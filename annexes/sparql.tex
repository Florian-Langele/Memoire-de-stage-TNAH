\label{sparql}
\begin{verbatim}
prefix oresm-onto: <http://data.oresm.fr/ontology#>
prefix rico: <https://www.ica.org/standards/RiC/ontology#>
prefix rdfs: <http://www.w3.org/2000/01/rdf-schema#>
prefix rdf: <http://www.w3.org/1999/02/22-rdf-syntax-ns#>
prefix xsd: <http://www.w3.org/2001/XMLSchema#>
PREFIX skos: <http://www.w3.org/2004/02/skos/core#>
PREFIX html: <http://www.w3.org/1999/xhtml>

#Récupère chaque rico:Person et les tris par nom de famille.

SELECT DISTINCT ?acteur ?prenom ?nomfamille (GROUP_CONCAT(DISTINCT 
?occupationl;separator='|') as ?occupations) (GROUP_CONCAT(?normalise;
separator = '\n') as ?toutes_dates) WHERE { 

    ?acteur rdf:type rico:Person.
    ?acteur rico:name ?nom.
    BIND(REPLACE(?nom,'.*([A-Z][a-àâçéèêëîïôûùüÿñæœ]+) ?(?:\\(|$|\\[).*','$1') 
    as ?nomfamille)
    BIND(REPLACE(?nom,'^(.+?) .+','$1') as ?prenom)
    
    OPTIONNAL{
    ?acteur rico:hasOrHadOccupationOfType ?occupation.
    ?occupation rdfs:label ?occupationl
    }
    
    ?record rico:isRelatedTo ?acteur; 
    rdfs:label ?nom_record ; 
    rdf:type ?record_ressource .
    ?record_ressource rdfs:subClassOf* rico:RecordResource.
   
    OPTIONNAL{
    ?record rico:isAssociatedWithDate ?date. 
    ?date rico:normalizedDateValue ?normalise
    }

}GROUP BY ?acteur ?prenom ?nomfamille ORDER BY ?nomfamille


---------------------------------------------------------

#Ajoute une relation rico:knows pour chaque personne qui est témoin du même acte. 


CONSTRUCT { ?personne1 rico:knows ?personne2 } WHERE { 
	?record oresm-onto:aPourTemoin ?personne1; 
            oresm-onto:aPourTemoin ?personne2.
    FILTER (?personne2 != ?personne1)
}


----------------------------------------------------------
#Compte le nombre de personne associé à une occupation.

SELECT (COUNT(?personne) as ?combien) ?occupationNom WHERE {
    ?personne rico:hasOrHadOccupationOfType ?occupation; rico:name ?noms.
    ?occupation rico:name ?occupationNom
} GROUP BY ?occupationNom

---------------------------------------------------------
\#Si le record a un testateur identifié alors c'est un testament


CONSTRUCT {
    ?testament 
    rico:isDocumentaryFormTypeOf 
    <http://data.oresm.fr/documentaryFormType/testament>
} WHERE { 
	?testateur oresm-onto:estTestateurDe ?testament.
}

---------------------------------------------------------
#Compte le nombre d'archives par langue et par siècle.

SELECT DISTINCT ?langueLabel ?siecle (COUNT(?record) as ?nombre) WHERE {
    ?record rico:hasOrHadLanguage ?langue; rico:isAssociatedWithDate ?date.
    ?langue skos:altLabel ?langueLabel.
    ?date rico:normalizedDateValue ?normalise.
    
    BIND(REPLACE(STR(YEAR(?normalise)+100),'[0-9]{2}$',' ème siècle') as ?siecle)
    
} GROUP BY ?langueLabel ?siecle ORDER BY ?siecle

---------------------------------------------------------
#Possible requête pour lier une femme ou une veuve à son mari
déjà présent dans le graphe identifié par le nom dans
le descriptive Note

PREFIX rico: <https://www.ica.org/standards/RiC/ontology#>
PREFIX rdf: <http://www.w3.org/1999/02/22-rdf-syntax-ns#>
PREFIX html: <http://www.w3.org/1999/xhtml>
CONSTRUCT {
    ?personne rico:hasOrHadSpouse ?acteur.
} where { 
    
	?personne a rico:Person ; 
        rico:descriptiveNote ?description; 
        rico:name ?nom_personne.
    ?acteur rdf:type rico:Agent; rico:name ?nom.
    BIND(REPLACE(str(?description),
    '[\\S\\s]+?(?=<html:p>)<html:p>([\\S\\s]+?) (\\[[^]]+\\])(?=<\\/html:p>)
    <\\/html:p>[\\S\\s]+','$1$2') as ?role).
    BIND(REPLACE(?role,"^[^A-Z]+(.*?)(?: et|,|\\[).*",'$1') as ?relatif).
    BIND(REPLACE(?role,"^([^A-Z]+).*?(?: et|,|\\[).*",'$1') as ?relation).
    
    FILTER(?relatif != '')
    FILTER(regex(?relatif,'[A-Z].*?[A-Z]'))
    FILTER(regex(?nom,?relatif))
    FILTER(regex(?relation,'femme|veuve'))
}
\end{verbatim}