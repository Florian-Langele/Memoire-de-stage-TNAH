\medskip
Ce mémoire a été rédigé à la suite d'un stage de quatre mois, dans le cadre du Master Technologies numériques appliquées à l’histoire de l’École nationale des chartes. Ce stage a été réalisé au sein du SERVAL (Service de la Valorisation numérique des collections et du soutien à la Recherche) sur le projet ORESM
(Œuvres et Référentiels des Étudiants, Suppôts et Maîtres). Le présent mémoire vise à exposer les réflexions et réalisations qu'ont entraînées le stage. Particulièrement en décrivant la production de métadonnées relatives à des archives médiévales et leurs sémantisations avec l'ontologie RiC-O. Le tout dans le cadre d'un projet de recherche ambitieux.

\bigskip
\textbf{Mots-clés }: Université de Paris; archives; RiC-O; Sparnatural ; graphe de connaissance ; web sémantique ; XSLT; RDF.

\bigskip
\textbf{Informations bibliographiques} : Florian Langelé, \textit{Production et exploitation d’un graphe de connaissances décrivant des archives médiévales : le cas du projet ORESM}, mémoire de master \og  Technologies numériques appliquées à l’histoire \fg  , dir. Florence Clavaud, École nationale des chartes, 2023.