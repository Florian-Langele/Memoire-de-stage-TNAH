Un projet de recherche passe nécessairement par de nombreuses étapes. De sa conception à sa réalisation, de l'idée qui germe de l'esprit à la dernière pierre posée il y aura toujours une multitude d'échanges, d'échecs, de remises en question et de réussites pour arriver finalement au résultat voulu. D'autant plus quand le projet innove sur un ou plusieurs aspects de son champ d'action. C'est le cas du projet ORESM conduit par la Bibliothèque Interuniversitaire de la Sorbonne (BIS).
\par
Dans ce mémoire nous présenterons les réflexions, décisions et réalisations qui ont eu lieu au cours du stage de quatre mois au sein du projet ORESM. Ce mémoire pourra permettre de guider la réflexion concernant le traitement et l'exploitation de données relatives à des archives médiévales. Il dresse une feuille de route rétrospective de la production à l'exploitation de ces données en s'appuyant sur l'expérience acquise durant le stage tout en englobant le propos dans le contexte du projet ORESM qui a ses enjeux bien particuliers. Pour résumer nous nous attacherons à expliquer comment les métadonnées archivistiques médiévales de forte granularité du projet ORESM ont été sémantisées afin d'être rendues exploitables par des chercheurs et étudiants.
\par
Nous débuterons ce mémoire par une partie présentant le contexte de notre stage : le projet ORESM, ses différents enjeux et l'avancement de celui-ci avec une présentation des différentes réalisations avant le début du stage. Suivra une partie qui présentera les données issues des dépouillements d'archives réalisés avant notre stage, le processus de sémantisation de ces données, tout en s'attachant à présenter les avantages et inconvénients du modèle suivi. En troisième partie, nous présenterons les premières utilisations que nous avons pu faire des données sémantisées : import dans une base de graphes, ébauche d'une interface de requête sur les données. Enfin, avant de conclure nous ferons un bref bilan du stage tout en présentant quelques pistes pour la poursuite du projet ORESM. Puisque le GitHub de travail du projet ORESM est privé, un \href{https://github.com/Florian-Langele/Memoire-de-stage-TNAH}{GitHub} a été créé pour stocker les livrables du stage associés à ce mémoire.