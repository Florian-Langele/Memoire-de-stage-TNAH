Finalement que pouvons nous tirer des réalisations de ce stage ? Tout d'abord la faisabilité de la chose. La granularité de nos données et leur complexité étaient les difficultés principales. Aucun autre projet utilisant RiC-O n'avait amené une description aussi précise d'archives médiévales, avec toutes les spécificités que cela implique. Il a fallu, pour opérer la transformation des données en fichiers tabulaires en un seul graphe de connaissance, faire un travail de réflexion sur leur sémantisation tout en prenant en compte les objectifs du projet. Il en ressort un processus de construction et de transformation des métadonnées clairement défini. Les méthodes employées sont documentées. L'exploitation bien que très partielle démontre de grandes possibilités. C'est encourageant pour la suite. Le projet s'inscrit dans la transition qui va arriver dans le monde des archives vers de nouvelles normes de description des métadonnées et peut s'enorgueillir d'être à la pointe en termes d'actualités technologiques. Mais il reste encore de nombreux chantiers avant d'arriver à un résultat publiable.
\par
Étant donné le temps disponible pendant notre stage tout ne pouvait pas être traité. Il a fallu mettre de côté certains aspects du projet, pour pouvoir produire à temps un résultat exploitable. A plusieurs reprises au cours du stage nous nous sommes fait la réflexion qu'avec un tel niveau de description nous pouvions passer des mois et des mois à parfaire nos données, à continuellement chercher à améliorer le modèle ou la transformation. Pourtant la meilleure manière pour progresser dans un projet est d'avancer, puis de revenir, si le besoin se fait sentir, sur une tâche antérieure, avec l'expérience acquise. Avec ce procédé nous avons une vision plus globale, ce qui éclaire la prise de décision. Nul doute que les éléments acquis à travers ce stage serviront de support aux différentes évolutions du projet.
\par
Par exemple, comme nous l'avons déjà évoqué, les remarques saisies dans les tableaux de dépouillement sont de nature très diverse. Ce qui est pratique et intellectuellement satisfaisant, lors d'une étape de saisie dans un tableau, n'est pas exploitable lorsqu'on s'attache à produire un graphe de haute qualité dans le cadre d'un projet de recherche. En partant de ce postulat, et avec le recrutement de nouveaux archivistes pour poursuivre les dépouillements, nous avons émis des recommandations afin de faciliter la transformation en RDF. Pour continuer sur cet exemple, il a été recommandé de typer les remarques\footnote{Voir le tableau rempli à cette occasion \href{https://github.com/Florian-Langele/Memoire-de-stage-TNAH/blob/main/Livrables/Typologie\%20des\%20remarques.xlsx}{Livrables/Typologie des remarques.xlsx}} par rapport à l'entité RiC-O auxquelles elles se rapportent. De manière plus générale, les données traitées lors du stage proviennent d'une seule archiviste, et évidemment en tant que humain celle-ci avait sa manière personnelle de saisir les données. Et même si elle suivait une méthodologie, elle avait sa logique propre sur laquelle se base le script de transformation. Avec l'arrivée de nouvelles personnes sur les dépouillements, les prochains mois permettront d'ajuster en parallèle, si nécessaire,  la méthodologie de saisie et le script de conversion.
\par
Il reste encore un travail à faire pour gérer la normalisation des noms de personnes et la réconciliation des entités nommées. Un point majeur du sujet prosopographique repose sur l'incertitude qu'on a concernant une personne apparaissant sur deux actes différents. Quand la personne occupe un rang important on sait beaucoup de choses sur elle et on peut s'y retrouver. Mais sur la majorité des personnes citées dans nos archives on ne sait rien de plus que ce que nous avons pu tirer des documents. D'autant plus que la période étudiée, l'époque médiévale, voit des formes de noms très variées. Parfois une même personne peut être identifiée dans plusieurs documents sous un même nom mais avec une graphie différente. Il est donc souvent impossible d'affirmer une identité entre deux personnes; on peut seulement spécifier une probabilité et tenter de quantifier le niveau d'incertitude qui en résulte. C'est l'objectif de la base de données Studium Parisiense de réunir toutes les informations sur les personnes liées à l'Université de Paris au Moyen Âge. La sémantisation de cette base et, plus largement, la réalisation d'un référentiel des personnes permettant d'opérer ces réconciliations en fonction de critères précis et de consigner un niveau d'incertitude, sont prévues dans le cadre du projet ORESM. Il conviendra de faire en sorte que ce référentiel et le graphe de connaissance que nous avons commencé à produire soient liés, voire échangent des données.
 
\par
Il faut également réfléchir à l'intégration de nos données dans l'inventaire virtuel. Une des pistes possibles serait d'utiliser RiC-O Converter sur les fichiers EAD ; cependant rien n'est décidé à ce stade. Une chose est certaine : il faudra intégrer au graphe les données qui concernent les ensembles d'archives dans lesquels les pièces décrites s'inscrivaient au Moyen Âge. L'expérience acquise sera essentielle pour assurer une fusion propre et efficace de ces données.
\par
Pour ce qui est de la préparation de l'interface Sparnatural, la configuration actuelle a été déployée sur un serveur interne à la BIS pour préparer une relecture et un recueil des besoins. Beaucoup d'autres choses restent à faire en ce qui concerne l'interface : Sparnatural est un des dispositifs, il faudra en mettre en place d'autres. Notamment pour afficher les données relatives à une unique entité. Ce stage n'était qu'une étape.