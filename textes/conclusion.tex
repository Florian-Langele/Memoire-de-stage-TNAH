Ce mémoire rend compte des différents processus conçus pour exploiter efficacement des données archivistiques précises. Au départ de ce stage, nous disposions de plusieurs fichiers tabulaires, homogènement structurés et décrivant des archives médiévales. En utilisant les bases, posées par la preuve de concept de Florence Clavaud, nous avons réalisé un traitement sur les données extraites de ces fichiers. Avec un unique script, nous avons automatisé la sémantisation de ces données au format RDF/XML. Ce script se base sur le \textit{data mapping}, qui fait correspondre chacun des champs définis par la méthodologie de dépouillement avec un élément de l'ontologie ORESM, qui étend l'ontologie RiC-O. Un début d'interface a été mis en place, en utilisant l'outil Sparnatural, pour permettre l'interrogation des données à l'aide d'un constructeur de requête visuel. Bien que les réalisations de ce stage soient concluantes, beaucoup de choses restes à faire avant leurs exploitations complètes pour répondre aux enjeux scientifiques du projet.
\par
Participer au projet ORESM a été particulièrement formateur. Par les différents interlocuteurs que ce projet réunit, il a permis d'en apprendre davantage sur plusieurs milieux, notamment le monde de la recherche, des bibliothèques et des archives, et évidemment sur le dialogue nécessaire pour faire travailler conjointement les différents acteurs issus de ces milieux. Quelques années vont encore être nécessaires pour réaliser l'ensemble des objectifs prévus, mais nul doute qu'une fois la plateforme ORESM achevée celle-ci aura une place particulière pour la recherche sur l'ancienne université de Paris et la valorisation de son patrimoine.